%%
%%  Department of Electrical, Electronic and Computer Engineering.
%%  EPR400/2 Final Report - Section 4.
%%  Copyright (C) 2011-2018 University of Pretoria.
%%

\section{Results}

\subsection{Summary of results achieved}

\begin{center}
	\begin{longtable}{|p{5cm}|p{5cm}|p{5cm}|}
		\hline
		\textbf{Description of requirement or specification (intended outcome)} &
		\textbf{Actual outcome} &
		\textbf{Location in report} \\
		\hline 
		\multicolumn{3}{|l|} {\textbf{Mission requirements of the product}} \\
		\hline
		Text recognition should
		consider the probability of
		occurance of combinations of
		letters.
		&
		The Viterbi algorithm was successfully implemented to detect the closest words which were in the user's dictionary.
		&
		Section 4.2.1
		\\
		\hline
		The user should be able to undo an input onto the touchscreen and not have the written sentence converted to text.
		&
		The device's GUI had a "Clear" button which had the capability of resetting the touchscreen to being blank when clicked.
		&
		Section 4.2.3\\
		\hline
		The system should recognize
		that the user has completed
		writing their shopping list.
		& 
		The device's software system correctly realized when user input was finished by keeping track of the "Finished" button on the GUI, which when switched the booolean variable that tracked if input was finished.
		&  Section 4.2.3\\
		\hline
		\multicolumn{3}{|l|} {\textbf{Field Conditions}} \\
		\hline
		The device should read handwriting input from multiple users.
		&
		The device could accurately recognize handwritings from multiple users. 
		&
		Section 4.2.2
		\\
		\hline
		The device should operate under varying light conditions that will be typical of what the user will encounter in indoor environments.
		&
		Operation of the device was executed in a dark room and in the presence of a bright fluorescent light. The device successfully carried out all its function in the two different light conditions.
		&
		Section 4.2.5
		\\
		\hline
		\multicolumn{3}{|l|} {\textbf{Specifications}} \\
		\hline
		Information should be sent at
		request, to the mobile phone
		within 10 seconds.
		&
		The measured time for the transmission time of the email containing the shopping list was found to have a mean of 8.35 s
		&   
		Section 4.2.2
		\\
		\hline
		Accuracy of the handwriting
		character detection must be
		greater than 80$\%$.
		&
		The average accuracy calculated from different unseen handwritings was found to be 83.3$\%$.
		&   Section 4.2.1\\
		\hline
		The user must be warned
		when the battery power
		remaining is less than 10$\%$ of
		full capacity.
		&
		An algorithm was programmed to continuously check the battery progress bar as a global variable of the project and when it reached less than $10\%$, a boolean flag for low battery was triggered.
		&	Section 4.2.3\\
		\hline
		Handwritten input must be converted to text by the character recognition software in less than 1s.
		&
		The average conversion time for a line sentence input was measured to be 690 m s, using the timer. 
		&
		Section 4.2.3\\
		\hline
		The communication link
		between the device’s
		communication system
		and the phone system should
		transfer data bidirectionally,
		at a rate of between 5Mbps
		and 10Mbps so that the bit
		error rate can be less than
		10$\%$.
		&
		The Wi-Fi link on the device could communicate at speeds of 11 Mbps.
		&
		Section 4.2.2
		\\
		\hline
		\multicolumn{3}{|l|} {\textbf{Deliverables}} \\
		\hline
		Optical character recognition algorithm for handwritten input to text conversion. The algorithm will be implemented on the hand-held shopping list device.
		&
		The optical recognition algorithm was successfully designed for and implemented onto the software system.
		&
		Section 4.2.1
		\\
		\hline
		Integration of the embedded processor
		board, the communication system and the touchscreen LCD to make the shopping list hardware device.
		&
		The different hardware was integrated to create the full working physical system which could be attached to a fridge.
		&
		Section 4.2.4 \\
		\hline	
		\caption{Results summary}
	\end{longtable}

\end{center}

\subsection{Qualification tests}

\subsubsection{Qualification test 1: Measurement of handwriting recognition accuracy}

4.2.1.a Qualification test\\
\textit{Objectives of test / experiment}\\
The accuracy of the handwriting scheme was measured in the test. The converted output was also checked against the dictionary and the relevant output was shown.

\textit{Equipment used}\\
The Python programming language was used. The mathematical libraries contained as well as the tracking libraries were used to develop an algorithm that could keep track of the recognized outputs and compare them to the expected values. The pseudo-code for the Python program implemented to calculate the accuracy of the recognition algorithm is shown below:\\

\begin{algorithmic}
	\STATE expectedOuputs.txt $\Leftarrow$ $\hfill$ $\rightarrow$ Text file containing expected text outputs.
	\WHILE {userInput == \textbf{true}}  
	\STATE \textbf{getUserLineInput()}  \hfill $\rightarrow$ Capture the user's handwritten shopping list line.
	\STATE sentence $\leftarrow$ \textbf{convertLineInput()} \hfill $\rightarrow$ Save recognition output
	\STATE dictionaryOut $\leftarrow$ Viterbi(sentence) \hfill $\rightarrow$ Detect closest word in dictionary.
	\STATE shoppingList.txt $\leftarrow$ dictionaryOutput \hfill Append the shopping list with the recognized sentence.
	\ENDWHILE
	\STATE correct = 0 \hfill $\rightarrow$ Correctly recognized letters
	\STATE total = 0 \hfill $\rightarrow$ Total recognized letters.
	\FOR {every line in shoppingList.txt}
	\FOR {every character i in every line}
	\IF {shoppingList.i == expectOutputs.i}
	\STATE correct $\leftarrow$ correct++ \hfill Increment correct characters
	\ENDIF
	
	total $\leftarrow$ total$++$ \hfill Increment total detected characters
	\ENDFOR
	\ENDFOR
	
	\STATE accuracy  $=$ $\frac{\text{correct}}{\text{total}} \times 100\%$
\end{algorithmic}

\textit{Experimental Parameters and setup}\\
The above values of accuracy were the important test parameters and the algorithm was implemented in Python.

\textit{Experimental protocol}\\
The basic steps in the calculation of the accuracy were extracted from the comments in the pseudo-code above:
\begin{itemize}
	\item Take the user input and convert to the corresponding sentence.
	\item Take the sentence and convert its contents to relevant dictionary words.
	\item Compare every letter from this output to the expect shopping list contents.
	\item Calculate output value.
\end{itemize}

4.2.1.b Results and Observations\\\\
\textit{Measurements}
\begin{center}
	\begin{longtable}{|p{5cm}|p{5cm}|}
		\hline
		\textbf{Number of sentences} &
		\textbf{Average Accuracy ($\%$)} \\
		\hline
		5
		&
		92.5
		\\
		\hline
		10
		&
		88.76
		\\
		\hline
		15
		&
		81.95
		\\
		\hline
		20
		&
		78.32
		\\
		\hline
		30
		&
		82.81
		\\
		\hline
		\textbf{Cumulative accuracy}
		&
		84.87
		\\
		\hline		
		\caption{Accuracy results}
	\end{longtable}
\end{center}

\begin{figure}[h]
	\centering
	\includegraphics[scale=0.4]{101}
	\caption{First line user input}
\end{figure}

\begin{figure}[h]
	\centering
	\includegraphics[scale=0.4]{102}
	\caption{Second line user input}
\end{figure}

\begin{figure}[h]
	\centering
	\includegraphics[scale=0.4]{103}
	\caption{Third line user input}
\end{figure}

\begin{figure}[h]
	\centering
	\includegraphics[scale=0.4]{105}
	\caption{Fifth line user input}
\end{figure}

\begin{figure}[h]
	\centering
	\includegraphics[scale=0.4]{106}
	\caption{Output text file containing shopping list}
\end{figure}
\clearpage
\textit{Description of results}\\
The accuracy of the recognition algorithm decreased as the number of sentences were doubled from 5 to 10. The accuracy decreased again as 5 more lines of user input were inserted. Addition of another 5 lines decreased the accuracy. Adding 10 more lines of user input led to an increase in the accuracy. The general accuracy was calculated to be 84.87$\%$

\subsubsection{Qualification test 2: Measurement of the transmission time for sending email}

4.2.2.a Qualification test\\
\textit{Objectives of test / experiment}\\
The time taken to send an email from the shopping list device to the user's email address. The wireless network's link capacity was also checked. 

\textit{Equipment used}\\
A Quartz stopwatch was used for timing.
\\

\textit{Experimental parameters and setup}\\
The value of $t_{transm}$ was the transmission time of note between the shopping list device sending the email and the message appearing on the user's inbox. The assumption was that the user had their e-mail service auto-refreshing at period of less than a second so that a new message was recognized immediately.

\textit{Experimental protocol}\\
The stages that were involved with the experiment were:
\begin{itemize}
	\item Reset the stopwatch to zero and open the email service on the user's phone.
	\item Run the mail algorithm code and simultaneously press the start button.
	\item Wait for the new email notification to pop up on the user's phone.
	\item  Stop the stopwatch when the email notification is observed by the user.
\end{itemize}

4.2.2.b Results and Observations\\
\textit{Measurements}\\
\begin{center}
	\begin{longtable}{|p{5cm}|p{5cm}|}
		\hline
		\textbf{File size (Bytes)} &
		\textbf{Transmission time (s)} \\
		\hline
		115
		&
		4.22
		\\
		\hline
		780
		&
		6.48
		\\
		\hline
		1200
		&
		8.24
		\\
		\hline
		5650
		&
		17.32
		\\
		\hline
		500
		&
		6.95
		\\
		\hline		
		\caption{Transmission time values.}
	\end{longtable}
\end{center}

\begin{figure}[h]
	\centering
	\includegraphics[scale=0.7]{110}
	\caption{Email received from the home shopping list device}
\end{figure}

\textit{Description of Results}\\
The first test was implemented for a file with 5 different line inputs and of size 115 Bytes. The transmission time was extremely small. A few more lines were added and the transmission time was fast as well, only slower by a few seconds. The lines of the text file were almost doubled and the transmission time increased by about a second. A bigger file i.e. an image was sent as the next argument and the transmission time was slower than the previous one by a second. A \textbf{.pdf} file was the file document sent by the e-mail system and although it was about 100 times larger than the previous file sent, it was only delivered in double the transmission time of the smaller file.

\subsubsection{Qualification test 3: System detection of user finishing handwriting process and handwriting conversion}

4.2.3.a Qualification test\\
\textit{Objectives of test / experiment}\\
The test involved the detection of the end of a user input process for their line sentence and the conversion of the line input in less than 1 second.

\textit{Equipment used}\\
The internal timing libraries of the Python environment were used. The Widget libraries were used to create GUI buttons which were kept track of by boolean variables. The pseudo-code for the conversion time calculation was shown below:\\

\begin{algorithmic}
	\STATE saveButton.clicked $\leftarrow$ \textbf{false} \hfill Initialize an unclicked button for save functionality  
	\WHILE {saveButton.clicked==\textbf{false}}
	\STATE drawn $\leftarrow$\textbf{false} userDrawing \hfill Model the input drawn
	\ENDWHILE
	\STATE $\rightarrow$ saveButton has now been clicked
	\IF {batteryBar $\leq 10\%$}
	\STATE warningonGUI() \hfill $\rightarrow$ Shows the warning sign on the GUI widget if battery low
	\ENDIF
	\STATE \textbf{startTimer()}
	\STATE convertInput \hfill $\rightarrow$ Apply Recognition algorithm
	\STATE \textbf{stopTimer()}
	\STATE $t_{reco}$ $\leftarrow$ timerValue	\hfill Recognition time value
\end{algorithmic}

\textit{Experimental Protocol}\\
The flow chart in Figure 36 shows the stages of the experimental protocol. 
%\newpage
\begin{figure}[h]
	\centering
	\includegraphics[scale=0.55]{50}
	\caption{Flow chart showing the experiment protocol}
\end{figure}

4.2.3.b Results and Observations\\
\textit{Measurements}\\
\begin{center}
	\begin{longtable}{|p{5cm}|p{5cm}|}
		\hline
		\textbf{Sentence number} &
		\textbf{Conversion time (s)} \\
		\hline
		1
		&
		0.43
		\\
		\hline
		2
		&
		0.88
		\\
		\hline
		3
		&
		0.32
		\\
		\hline
		4
		&
		0.65
		\\
		\hline
		5
		&
		0.49
		\\
		\hline
		\textbf{Average Recognition Time}
		&
		0.55
		\\
		\hline		
		\caption{Recognition time}
	\end{longtable}
\end{center}

\begin{figure}[h]
	\centering
	\includegraphics[scale=0.5]{107}
	\caption{Button initialization on GUI}
\end{figure}

\textit{Description of Results}\\
The save button was initialized in the empty GUI before any input. The conversion time for the recognition time was randomly distributed for different lines. The conversion time increased when the second text line was input. However, the conversion time slipped to an even lower value for the 3$^{rd}$ sentence. The conversion time increased again for the forth line input before decreasing for the fifth. The mean recognition time for the device was calculated to be 0.55 s.

\subsubsection{Qualification test 4: Integration of device hardware}

4.2.4.a Qualification test\\
\textit{Objectives of test / experiment}\\
The experiment involved the integration of all the modulated hardware components into one fully-functional device.

\textit{Equipment Used}\\
Dupont wires were used for connection of the touchscreen and the Raspberry Pi development board. Male to female and female to female jumper cables were also used to connect the power inputs together. The beam balance was used to calculate the overall weight of the device. The Vernier calipers was used to calculate the length, width and how broad the integrated hardware was. 

\textit{Experiment Protocol}\\
The following stages were taken when the user integrated the full working product:
\begin{itemize}
	\item  Connect the power pin of the Raspberry Pi GPIO to the power pin of the touchscreen 
	\item  Connect the display module on the copper strip from the touchscreen to the open slot on the Raspberry Pi for the display.
	\item Insert the 3D printed plastic cover for the device with a removable back for Raspberry Pi debugging.
	\item Connect magnet to the backside of the integrated device.
	
\end{itemize}

4.2.4.b Results and Observations\\
\textit{Measurements}\\
\begin{center}
	\begin{longtable}{|p{5cm}|p{5cm}|}
		\hline
		\textbf{Measurable parameter} &
		\textbf{Value} \\
		\hline
		Mass
		&
		
		\\
		\hline
		Width
		&
		
		\\
		\hline
		Length
		&
		
		\\
		\hline		
		\caption{Fully integrated hardware parameters}
	\end{longtable}
\end{center}
//Insert pictures here

\textit{Description of results}\\
The hardware was successfully integrated. Power was supplied to the device by either powering the Raspberry Pi or the touchscreen since they shared the connection. The device was successfully stuck on a fridge for testing.

\subsubsection{Qualification test 5: Operation in dark light conditions}
4.2.5.a Qualification test\\
\textit{Objectives of test / experiment}\\
The test involved the determination of successful operation of the device at various light intensities.

\textit{Equipment used}\\
A cellphone was used to capture photos of the operation of the device at different light intensity.

\textit{Equipment protocol} \\
The following stages were sequentially taken for the test to be successful.
\begin{itemize}
	\item  Calculate the area of the room that the home shopping list device is operating on.
	\item Perform normal operation of the device outside the room.
	\item Send shopping list text to user'e email.
	\item  Switch off the room light
	\item Attempt to write onto the device and save it.
	\item  Request email of the shopping list from the user's phone.
\end{itemize}

4.2.4.b Results and Observations\\
\textit{Measurements}\\
//Insert pictures here

\textit{Description of results}\\
The device operated equally well lights were off in the operating room. The operation outdoors was also successful with the minor hindrance that the user could not perfectly see their touchscreen contents due to the reflection of the sunlight from the touchscreen.

\newpage

%% End of File.


